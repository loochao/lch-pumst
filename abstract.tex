<<<<<<< HEAD
A strong femto-second laser pulse ($800nm$, $10^{15} W/cm^2$, $100fs$ pulse width) was directed to Helium gas (atom density:$10^{18}cm^{-3}$). The atoms are fully ionized in this optical field (E: $10^8 V/m$), and plasma is created. Due to collisons, ionized electrons repopulate all the energy levels of Helium in nanoseond time scale. Then with beams from OPA directed to the gas, whose wavelengh matches the energy difference of levels intersted, the coherence in the atoms are created. With the presence of coherence,the lasing without inversion scheme is created. Finally a transint lasing gain without population inversion between $2^1P$ and $2^1S$ was expected, whose wavelength is $58nm$.
=======
This is a \LaTeX{} template and document class for Ph.D. dissertations at Princeton University. It was created in 2010 by Jeffrey Dwoskin, and adapted from a template provided by the math department. Their original version is available at: \url{http://www.math.princeton.edu/graduate/tex/puthesis.html}

This is \textbf{NOT} an official document. Please verify the current Mudd Library dissertation requirements~\cite{mudd2009} and any department-specific requirements before using this template or document class.


Your abstract can be any length, but should be a maximum of 350 words for a Dissertation for ProQuest's print indicies (or 150 words for a Master's Thesis); otherwise it will be truncated for those uses~\cite{proquest2006}.


Dwoskin Ph.D. Dissertation Template --- version 1.0, 5/19/2010
>>>>>>> parent of e75694e... Template only
